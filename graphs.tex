\documentclass{article}
\usepackage[utf8]{inputenc}
\usepackage{tikzit}
\usepackage{booktabs}

%%%%%%%%%%%%%%hyperlink %%%%%%%%%%%%%
\usepackage{hyperref}
\hypersetup{
    colorlinks=true,
    linkcolor=blue,
    filecolor=magenta,      
    urlcolor=cyan,
}
 
\urlstyle{same}
%%%%%%%%%%%%%%%%%%%%%%%%%%%%%%%%%%%%%

%%%%%%% Algorithm package %%%%%%%%%%
\usepackage{algorithm, algpseudocode}
\algdef{SE}[SUBALG]{Indent}{EndIndent}{}{\algorithmicend\ }%
\algtext*{Indent}
\algtext*{EndIndent}
%%%%%%%%%%%%%%%%%%%%%%%%%%%%%%%%%%%%%

%%%%%%% Mdframed settings %%%%%%%%%%%
\usepackage[framemethod = TikZ]{mdframed}
\input{sample.tikzstyles}
\mdfsetup{skipabove = \topskip, skipbelow = \topskip}
\global\mdfdefinestyle{exampledefault}{%
    outerlinewidth = 1pt, innerlinewidth = 0pt,
    outerlinecolor = blue, roundcorner = 5pt
}
%%%%%%%%%%%%%%%%%%%%%%%%%%%%%%%%%%%%%

%%%%%%% Theorems, Proofs %%%%%%%%%%%%
% ref: https://www.overleaf.com/learn/latex/Theorems_and_proofs
\usepackage{amsthm}
\newtheorem{theorem}{Theorem}[section]
\newtheorem{observation}{Observation}[section]
\newtheorem{corollary}{Corollary}[theorem]
\newtheorem{lemma}[theorem]{Lemma}
% for adding unnumbered theorems
% \theoremstyle{remark}
\newtheorem*{remark}{Remark}
\theoremstyle{definition}
\newtheorem{definition}{Definition}[section]
%%%%%%%%%%%%%%%%%%%%%%%%%%%%%%%%%%%%%

%%%%%%% Math and Symbols %%%%%%%%%%%%
\usepackage{amsmath, amssymb}
%%%%%%%%%%%%%%%%%%%%%%%%%%%%%%%%%%%%%

%%%%%%%%%%%%%  Macros  %%%%%%%%%%%%%%
\newcommand{\true}{\texttt{TRUE}}
\newcommand{\false}{\texttt{FALSE}}
%%%%%%%%%%%%%%%%%%%%%%%%%%%%%%%%%%%%%


\title{2-SAT and Biconnected Components}
\author{bhavit.sharma804 }
\date{\today}

\begin{document}
\maketitle
\tableofcontents

\section{2-sat}
\subsection{Introduction}
\paragraph{} Consider a boolean function
\[
	f = (x_1 \lor y_1)\land(x_2 \land y_2) \cdots (x_n \lor y_n)
\]

\par \textit{Is $f$ satisfiable?}
To make $(a \lor b) = 1$, we need to consider two cases:
\begin{itemize}
	\item if $a = 0$, then $b$ must be either $1/0$. Hence $-a \implies b$
	\item if $b = 0$, then $a$ must be either $1/0$. Hence $-b \implies a$
\end{itemize}

Hence, for every clause $a \lor b$, we will add two clauses $(-a \implies b)$ and $-b \implies a$
in the graph. For example, if our clause
\[
    \phi = (-x \lor y) \land (-y \lor z) \land (x \lor -z) \land (z \lor y)
\]

Hence, our 2-CNF graph will look like this.\\
\ctikzfig{fig1}

\subsection{Solution}
Let's say we have $n$ clauses in our boolean function. The total number of vertices and edges in
graph are $\mathcal{O}(n)$. Some observations which might help us.

\begin{observation}
    \label{Obs1}
    If there exists a path from node $i$ to $j$, then there exists a path from node
    $-j$ to $-i$.
\end{observation}

\begin{proof}
    This is because whenever we insert $a \implies b$ to the set, we also insert
    $-b \implies -a$ in the set. Hence, if there exists a path $(a, a_1, a_2, \cdots, b)$
    then, the path $(-b, \cdots, -a_2, -a_1, -a)$.
\end{proof}

\begin{observation}
    \label{Necessary}
    If there exists a path from $x$ to $-x$ and path from $-x$ to $x$, then
    boolean function is not satisfiable.
\end{observation}

\begin{proof}
    Let us assign the value of \texttt{TRUE} to $x$. Since, a path from $-x$ exists,
    we will get the value of $-x$ to be \texttt{TRUE} which is a contradiction.
    If we assign the value of $x$ to be \texttt{FALSE}, we again get a contradiction,
    since path from $-x$ to $x$ exists. 
\end{proof}

\begin{observation}
    As it turns out, Observation \ref{Necessary} is both Necessary and Sufficient.
\end{observation}

\begin{proof}
    To begin the proof of the \textit{sufficient} condition, we will present a constructive
    algorithm and then analyse its correctness.
\end{proof}
    \begin{mdframed}[frametitle = Algorithm, style = exampledefault]
        \begin{itemize}
            \item Create the directed graph and decompose it into Strongly Connected Components(SCC).
            \item Topological sort the SCCs and create an array \texttt{comp[i]} denotes the position of node's SCC in top-sort of the SCCs.
            \item If \texttt{comp[i] < comp[-i]}, then we assign $i =$ \texttt{FALSE} else \texttt{TRUE} otherwise.
            \item If \texttt{comp[i] == comp[-i]}, then it's not satisfiable.
        \end{itemize}
    \end{mdframed}
    Now, let us consider the case when $x = $ \true\ i.e. \texttt{comp[-i] < comp[i]},
    which means that $x$ and $-x$ \textbf{doesn't lie} in the same component.

\begin{theorem}
    There doesn't exists a node $y$ such that $-x$ has a path to $y$ and $-y$, both.
\end{theorem}

\begin{proof}
    We'll prove this by contradiction. If $y, -y$ are reachable from $-x$, then $x$ is reachable from $-y, y$
    (By Observation \ref{Obs1}). Hence, $x$ is reachable from $-x$, and $-x$ is reachable from $x$, which means
    that they lie in the same component (a contradiction).
\end{proof}

We have constructed an algorithm which find the correct value of the constraints,
under the assumption that $x$ and $-x$ lie in different components. Hence, we have proved that the Observation \ref{Necessary}
is both Necessary and Sufficient.

\subsection{Conversion of Two-variable Boolean Function}
Till now we've seen the clauses to be boolean function of \textsc{logical or}, but the
clauses can be any function, like \textsc{xor, nand, nor} etc. For example our boolean function
might look like this.
\[
    f = (x_1 \text{xor} x_2) \land (-x_1 \lor x_3) \land (-x_3 \text{nand} x_1)
\]
To convert any boolean(2 variables) function into an equivalent 2-CNF form, we'll perform the
\textit{Products of Sum} transformation to the truth table of the boolean function. For example, the
truth table of \textsc{xor} is:

\begin{center}
\begin{table}[htb]
\begin{tabular}{|l|l|l|}
\hline
0 & 1 & 1 \\ \hline
0 & 0 & 0 \\ \hline
1 & 1 & 0 \\ \hline
1 & 0 & 1 \\ \hline
\end{tabular}
\end{table}
\end{center}

So, its POS is $(a + -b)(-a + b) = (a \lor -b) \land (-a \lor b)$.

\subsection{Implementation}
\href{https://github.com/Equlnox/IITP-ACM-Notebook/blob/master/code/graphs/2sat.cpp}{Link to Code}. The code
is implemented in such a way that you just have to provide the truth table of the boolean function to class, and it'll handle all of the stuff.
Important thing to keep in mind is, it's $1$ based indexing. To get the value of actual assignment,
just look at the \texttt{std::vector comp[i]}

\subsection{Problems}
TODO

\section{Biconnected Components}

\end{document}